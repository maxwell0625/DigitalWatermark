% -*- coding: UTF-8 -*-
%!TEX program = xelatex
\documentclass[a4paper,zihao=5,UTF8]{ctexart}

% 设置section标题居左
\CTEXsetup[format={\heiti\zihao{3}\bfseries}]{section}
\setcounter{section}{-1}
\CTEXsetup[format={\heiti\zihao{-3}\bfseries}]{subsection}
\usepackage{array}
\usepackage{multicol}
%数学包,这里没用到
%\usepackage{amsmath}
\usepackage{indentfirst}
%添加作者信息
\usepackage{authblk}
\usepackage{graphicx}

% 三线表
\usepackage{booktabs}
\usepackage{longtable}
\usepackage{lscape}

\usepackage{booktabs}
%设置标题字体,因为section一般为粗体。
\usepackage{fontspec}
\setCJKmainfont[BoldFont=KaiTi]{SimSun}

\usepackage{titlesec}
\usepackage{float}
\usepackage{appendix}
% \newfontfamily\sectionef{heiti}
\usepackage{graphicx}
\graphicspath{ {./figures/} }
\titleformat*{\section}{\zihao{3}\heiti}
\titleformat*{\subsection}{\zihao{-3}\bfseries\heiti}
\titleformat*{\subsubsection}{\large}

%页码格式
\pagestyle{plain}
%设置书签	
\usepackage[bookmarks=true,colorlinks,linkcolor=black]{hyperref}

% \usepackage[dvipsnames]{xcolor}%颜色宏包
\usepackage{listings}

% \lstset{
%     language = Python,
%     backgroundcolor = \color{yellow!10},    % 背景色:淡黄
%     basicstyle = \small\ttfamily,           % 基本样式 + 小号字体
%     rulesepcolor= \color{gray},             % 代码块边框颜色
%     breaklines = true,                  % 代码过长则换行
%     %numbers = left,                     % 行号在左侧显示
%     %numberstyle = \small,               % 行号字体
%     keywordstyle = \color{blue},            % 关键字颜色
%     commentstyle =\color{green!100},        % 注释颜色
%     stringstyle = \color{red!100},          % 字符串颜色
%     frame =  tb,                  % 用(带影子效果)方框框住代码块
%     showspaces = false,                 % 不显示空格
%     columns = fixed,                    % 字间距固定
% }

%英文摘要
\newcommand{\enabstractname}{}
\newenvironment{enabstract}{%
\quotation
	\par\small
	\mbox{}\hfill{\bfseries \enabstractname}\hfill\mbox{}\par
	\vskip 2.5ex}{\par\vskip 2.5ex} 

\title{\Large\CJKfamily{zhkai} plaintext}	
\author{
        \zihao{-4}
        辛正非$^2$\quad 
    }
\affil{(
    \zihao{-5}
    {上海第二工业大学\quad{}上海\quad{}201209} 
)}					   								
\date{}

% 页边距
\usepackage{geometry}
\geometry{a4paper, scale=0.9}

\begin{document}
	\maketitle
	{
		% \footnotetext[1]{清华大学在读研究生。
		% }
	}
    
    \noindent\zihao{-5}\heiti{%不平衡双栏
        \>摘要:
    }
    \zihao{-5}\songti{
        plaintext
    }

    \noindent\zihao{-5}\heiti{
        关键字:
    }
    \zihao{-5}\songti{
        
    }

\section{概述}

\subsection{数字水印}

\subsection{评价方式}

\subsection{实现方式}

\section{算法实现}

\subsection{最低有效位算法}

\subsection{随机间隔算法}

\subsection{区域校验位算法}

\section{结果分析与对比}

\section{基于RPC提供加解密服务}

\section{总结}

\begin{thebibliography}{10}

% \bibitem{ref1}
% 窦茂伟,张宝明. 电商平台退货的动态博弈分析[J]. 电子商务,2019,(06):47-48.

% \bibitem{ref2}
% 张璨璨. 服装业逆向物流的现状及对策分析[J]. 商场现代化,2019,(10):55-56.

% \bibitem{ref3}
% 姬雅帅. 关于逆向物流的博弈分析[J]. 农业装备与车辆工程,2017,55(08):84-87.

% \bibitem{ref4}
% 尚月. 电商环境下逆向物流存在的问题及解决方案[J]. 农村经济与科技,2017,28(14):76+78.

% \bibitem{ref5}
% 王文哲,杨锐锐. B2C环境下退货逆向物流研究[J]. 湖北工程学院学报,2016,36(05):102-105.

\end{thebibliography}

\begin{center}
    \Large{
        \textbf{
            Title in English
        }
    }

    \zihao{5}{
        Zhengfei Xin\quad 
    }
    
    \zihao{-5}{
        \emph{
            Shanghai Polytechnic University\quad{}Shanghai\quad{}201209
        }
    } 
\end{center}

\noindent\zihao{-5}\heiti{
    \textbf{Abstract:}
}
\zihao{-5}\songti{

}

\noindent\zihao{-5}\heiti{
    \textbf{Keywords:}
}
\zihao{-5}\songti{
    
}
\end{document}
